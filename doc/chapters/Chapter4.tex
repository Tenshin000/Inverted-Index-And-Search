\section{MapReduce and Hadoop code}
The system uses \textbf{MapReduce}, via the \textbf{Hadoop} framework, to process large-scale data efficiently. The Hadoop cluster is optimized for virtual machines with limited memory through customized YARN and MapReduce settings. YARN manages resources and memory (up to 5 GB per node), while MapReduce configurations allocate 2048 MB to key tasks, with JVM heaps limited ot 1536 MB.

\newpage 

%PseudoCode
\lstdefinelanguage{PseudoCode}{
	keywords={class, method, for, all, in, do, end, Emit, if, else, then},
	sensitive=true,
	morecomment=[l]{//},
	morecomment=[s]{/}{/},
	morestring=[b]",
	keywordstyle=\bfseries\color{blue},
	commentstyle=\color{gray}\itshape,
	stringstyle=\color{red},
	basicstyle=\ttfamily\small,
	numbers=left,
	numberstyle=\tiny\color{gray},
	stepnumber=1,
	numbersep=5pt,
	showspaces=false,
	showstringspaces=false,
	tabsize=2,
	breaklines=true,
	breakatwhitespace=true,
	escapeinside={(@}{@)}
}
\begin{center}
	\begin{minipage}{0.9\linewidth}
		\begin{lstlisting}[language=PseudoCode]
class TokenizerMapperStateful

	method initialize()
		word_counts <= New Empty AssociativeArray()
	end method
	
	method map(offset o, doc d)
		Filename <= retrieve_file_name()
		
		for all term t in doc d do
			if word_counts[t] does not contain Filename then
				word_counts[t][Filename] <= 1
			else
				word_counts[t][Filename] <= word_counts[t][Filename] + 1
			end if
		end for
		
		if word_counts.size() > FLUSH_THRESHOLD then
			flush(context)
		end if
	end method
	
	method flush(context)
		for each word in word_counts do
			for each filename in word_counts[word] do
				value <= filename + ":"+ 
				+word_counts[word][filename]
				
				emit(word, value)
			end for
		end for
		clear word_counts
	end method
	
	method cleanup(context)
		flush(context)
	end method

end class
		\end{lstlisting}
		\captionof{figure}{Stateful Mapper PseudoCode}
		\label{fig:pseudocode-stateful-mapper}
	\end{minipage}
\end{center}
\begin{center}
	\begin{minipage}{0.9\linewidth}
		\begin{lstlisting}[language=PseudoCode]
class TokenizerMapper
	method map(offset o, doc d)
		Filename <= retrieve_file_name()
		for all term t in doc d do
			emit(term t, filename + ":1")
		end for
	end method
end class
		\end{lstlisting}
		\captionof{figure}{Stateless Mapper PseudoCode}
		\label{fig:pseudocode-stateless-mapper}
	\end{minipage}
\end{center}
	\begin{center}
		\begin{minipage}{0.9\linewidth}
			\begin{lstlisting}[language=PseudoCode]
class DocumentCountReducer
	method reduce(term, postingsList)
		docCounts <= {}
		for all posting in postingsList do
			for pair in split(posting, ",") do
				doc, cnt <= splitLast(pair, ":")
				docCounts[doc] docCounts.get(doc, 0) + toInt(cnt)
			endfor
		endfor
	emit(term, format(docCounts))
	end method
end class
			\end{lstlisting}
			\captionof{figure}{Reducer PseudoCode}
			\label{fig:pseudocode-reducer}
		\end{minipage}
\end{center}

The application offers two interchangeable MapReduce variants: a classic \textbf{Mapper with optional Combiner} and a \textbf{Stateful In-Mapper Combiner}, selectable via command-line flags for modularity and flexibility.

To address Hadoop's inefficiency with many small files, we use \texttt{CombineFileSplit} to reduce task overhead. 

In Hadoop, the \texttt{CombineFileInputFormat} class does not know by itself how to read each individual file within a \texttt{CombineFileSplit}. For this reason, it requires a \textbf{custom RecordReader} for each combined file. This is the job of \texttt{MyCombineTextInputFormat}.

\texttt{MyCombineFileRecordReaderWrapper} is a wrapper around LineRecordReader, which reads one line at a time as in a normal Hadoop job. Its main function, however, is another: \textbf{it keeps track of the name of the file it is reading from}, using a \texttt{ThreadLocal} object. This is essential for an inverted index, because each word read from the line must be associated with the document (i.e. the file) in which it appears. 

The \texttt{TokenizerMapStateful} class accumulates word counts in memory using a data structure initialized in \texttt{setup()}, mapping words to document-specific counts. Once a predefined \textbf{threshold} is exceeded, it triggers a \textbf{flush}, emitting partial results in the format: $\langle \textbf{word}, \textbf{doc-id:count} \rangle$

Residual data is emitted during \texttt{cleanup()}.

In contrast, the \texttt{TokenizerMapper} class (lacking in-mapper combining) directly emits key-value pairs of the form:
\[
\langle \textbf{word}, \textbf{doc-id:1} \rangle
\]

The \texttt{CombinerDocCounts} class implements the Combiner logic, aggregating intermediate values by summing occurrences per document:
\[
\langle \textbf{word}, \textbf{doc-id:count} \rangle
\]

Finally, the \texttt{DocumentCountReducer} consolidates all partial counts per word across files and formats the output as:

\[
\texttt{word \textbackslash t filename1:count1 \textbackslash t filename2:count2}
\]
