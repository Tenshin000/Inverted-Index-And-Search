\section{Non‑parallel code}
The non‑parallel Python implementation builds an inverted index on a single node using a SPIMI (Single Pass In‑Memory Indexing) approach. It parses command‑line arguments for local or HDFS input/output URIs, optional size limits, and verbosity via \texttt{argparse}. Text files are read (either from HDFS via \texttt{hdfs.InsecureClient} or from the local filesystem), normalized (punctuation and underscores replaced by spaces), tokenized, and accumulated in an in‑memory index. Every \texttt{BLOCK\_SIZE} files—or when the size limit is reached—the current postings are flushed to a sorted block file in a temporary directory.  
After all blocks are written, a multi‑way merge using a min‑heap combines block files into a single output (written back to local or HDFS), preserving term order and aggregating postings. The script prints a concise summary of block construction time, merge time, total runtime, and memory usage.  
