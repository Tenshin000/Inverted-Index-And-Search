\section{Search Query System}
The \texttt{search\_query.py} utility offers a unified CLI for querying inverted indexes generated by local, Hadoop or Spark runs. It defines mutually exclusive flags (\texttt{--folder}, \texttt{--hadoop}, \texttt{--spark}, \texttt{--spark‑rdd}, \texttt{--non‑parallel}) via \texttt{argparse}. Depending on the flag, it loads lines from local files or HDFS (\texttt{hdfs.InsecureClient}), then invokes \texttt{build\_term\_offset\_index\_from\_lines} to map each lowercase term to its file‑line offset. During interactive querying, input terms are normalized, their postings lists retrieved by offset, and filename sets intersected to yield a sorted result list. The REPL loops until Ctrl+D, printing “No matches found.” if empty.  
